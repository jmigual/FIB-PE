\documentclass[a4paper, 12pt]{article}

\usepackage[dvipsnames]{xcolor} % Code highlighting color
\usepackage{fontspec} 
\usepackage[hidelinks]{hyperref} % Links color
\usepackage[catalan]{babel} % Language 
\usepackage{import}
\usepackage{listings} % Add code
\usepackage{fullpage}
\usepackage[a4paper, margin=2cm]{geometry} % To change the margins
\usepackage{graphicx} % Insert images
\usepackage{pdfpages}
\usepackage{ragged2e}
\usepackage{wrapfig} %To Text wrap around figures

\setlength\parindent{0pt}

\begin{document}

\title{Estudi de la ocupació de les aules de la FIB}
\author{Arnau Canyadell Miquel \and Joan Marcè Igual \and Daniel}

\maketitle
\newpage
\tableofcontents

\newpage
\section{Introducció}

A la FIB és molt comú anar en busca d'una aula d'informàtica per tal de fer un treball o altre. A nosaltres ens va picar la curiositat per saber quina era la ocupació mitjana d'aquestes aules i a quines hores i quins dies estaven menys ocupades per tal de poder trobar-ne una lliure amb facilitat. Evidentment aquestes dades només serviran per aquest quadrimestre ja que cada quadrimestre les assignatures que es fan a la FIB canvien el seu nombre d'alumnes matriculats i les aules on es realitzen.

\subsection{Objectiu}

El nostre treball consisteix en analitzar la ocupació de les aules d'ordinadors de la \emph{Facultat d'Informàtica de Barcelona}. Volem saber quant ocupades estan les aules i com es distribueix aquesta ocupació al llarg dels dies de la setmana.

La nostra hipòtesi és \emph{En les aules on hi ha classe la ocupació és menor que en les que hi ha classe}

\section{Metodologia}

\subsection{Recollida de dades}
Per a fer la recollida de dades, hem programat un servidor perquè es connecti a l'API de la FIB cada minut des del dimarts 5 de maig fins el divendres 29 del mateix mes i des de les 8 del matí fins a les 9 del vespre (horari d'obertura de la facultat). Degut a un error en l'execució del programa, entre els dies 5 i XX, només disposem de dades a partir de les  11:44.
L'API del racó ens proporciona la informació següent:
\begin{itemize}
	\item L'horari de classes de cada aula.
	\item El nombre d'ordinadors lliures per cada aula en l'instant en què es demana.
\end{itemize}

\subsection{Variables d'estudi}
Per tal de fer-nos una idea de com és l'ocupació de les aules, hem decidit treballar amb les variables següents:
\begin{description}
	\item[X] = proporció d'ordinadors lliures en les aules on \emph{hi ha classe}.
	\item[Y] = proporció d'ordinadors lliures en les aules on \emph{no hi ha classe}.
\end{description}

 
\subsection{Contrast d'hipòtesis}
$$H0: \mu_x = \mu_y$$
$$H1: \mu_x \neq \mu_y$$

\subsection{Estadístic utilitzat}
$$\hat{z} = \frac{\bar{x} - \bar{y}}{\sqrt{s_x^2/n_x + s_y^2/n_y}} $$





\end{document}