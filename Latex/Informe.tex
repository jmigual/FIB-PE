\documentclass{article}

\usepackage[dvipsnames]{xcolor} % Code highlighting color
\usepackage{fontspec} 
\usepackage[hidelinks]{hyperref} % Links color
\usepackage[catalan]{babel} % Language 
\usepackage{import}
\usepackage{listings} % Add code
\usepackage{fullpage}
%\usepackage[a4paper, margin=2cm]{geometry} % To change the margins
\usepackage{graphicx} % Insert images
\usepackage{pdfpages}
\usepackage{ragged2e}
\usepackage{wrapfig} %To Text wrap

\begin{document}

\title{Estudi de la ocupació de les aules de la FIB}
\author{Arnau Canyadell Miquel \and Joan Marcè Igual \and Daniel}

\maketitle
\newpage
\tableofcontents

\newpage
\section{Introducció}

A la FIB és molt comú anar en busca d'una aula d'informàtica per tal de fer un treball o altre. A nosaltres ens va picar la curiositat per saber quina era la ocupació mitjana d'aquestes aules i a quines hores i quins dies estaven menys ocupades per tal de poder trobar-ne una lliure amb facilitat. Evidentment aquestes dades només serviran per aquest quadrimestre ja que cada quadrimestre les assignatures que es fan a la FIB canvien el seu nombre d'alumnes matriculats i les aules on es realitzen.

\subsection{Objectiu}

El nostre treball consisteix en analitzar la ocupació de les aules d'ordinadors de la \emph{Facultat d'Informàtica de Barcelona}. Volem saber 

\end{document}