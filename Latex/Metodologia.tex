\section{Metodologia}

\subsection{Recollida de dades}
Per a fer la recollida de dades, hem programat un servidor perquè es connecti a l'API de la FIB cada minut des del dimarts 5 de maig fins el divendres 29 del mateix mes i des de les 8 del matí fins a les 9 del vespre (horari d'obertura de la facultat). Degut a un error en l'execució del programa, entre els dies 5 i XX, només disposem de dades a partir de les  11:44.
L'API del racó ens proporciona la informació següent:
\begin{itemize}
	\item L'horari de classes de cada aula.
	\item El nombre d'ordinadors lliures per cada aula en l'instant en què es demana.
\end{itemize}

\subsection{Variables d'estudi}
Per tal de fer-nos una idea de com és l'ocupació de les aules, hem decidit treballar amb les variables següents:
\begin{description}
	\item[X] = proporció d'ordinadors lliures en les aules on \emph{hi ha classe}.
	\item[Y] = proporció d'ordinadors lliures en les aules on \emph{no hi ha classe}.
\end{description}

 
\subsection{Contrast d'hipòtesis}
$$H0: \mu_x = \mu_y$$
$$H1: \mu_x \neq \mu_y$$

\subsection{Estadístic utilitzat}
$$\hat{z} = \frac{\bar{x} - \bar{y}}{\sqrt{s_x^2/n_x + s_y^2/n_y}} $$



