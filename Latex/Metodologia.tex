\section{Metodologia}

\subsection{Recollida de dades}
Per a fer la recollida de dades, hem programat un servidor perquè es connecti a l'API de la FIB cada minut des del dimarts 5 de maig fins el divendres 29 del mateix mes i des de les 8 del matí fins a les 9 del vespre (horari d'obertura de la facultat). Degut a un error en l'execució del programa, entre els dies 5 i XX, només disposem de dades a partir de les  11:44.
L'API del racó ens proporciona la informació següent:
\begin{itemize}
	\item L'horari de classes de cada aula.
	\item El nombre d'ordinadors lliures per cada aula en l'instant en què es demana.
\end{itemize}

\subsection{Variables d'estudi}
Per tal de fer-nos una idea de com és l'ocupació de les aules, hem decidit treballar amb les variables següents:
\begin{description}
	
	\item[X] = "nombre d'ordinadors situats en una aula on HI HA CLASSE que no estan utilitzats"/"nombre d'ordinadors situats en una aula on HI HA CLASSE"
	\item[Y] = "nombre d'ordinadors situats en una aula on NO HI HA CLASSE que no estan utilitzats"/"nombre d'ordinadors situats en una aula on NO HI HA CLASSE
\end{description}

 
\subsection{Anàlisi d'estadístics}
El comportament de les variables \emph{X} o \emph{Y} pot dependre de molts factors. Tot seguit n'enumerem uns quants:
\begin{description}
	\item[Hora:] els hàbits rutinaris dels estudiants i els horaris de classe fan que en certes hores hi hagi més gent utilitzant els ordinadors.
	\item[Ocupació de les aules per fer classe:] com més aules estiguin reservades per fer-hi classe, menys ordinadors lliures hi haurà pels estudiants que no tenen classe i en volen fer servir un.
	\item[Exàmens:] els exàmens amb ordinador, a part d'ocupar moltes aules, poden provocar que hi hagi molts estudiants repassant a l'últim moment i que l'ocupació en les hores abans sigui anormalment alta.
\end{description}



