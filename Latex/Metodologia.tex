\section{Metodologia}

\subsection{Recollida de dades}
Per a fer la recollida de dades, hem programat un servidor perquè es connecti a l'API de la FIB cada minut des del dimarts 5 de maig fins el divendres 29 del mateix mes i des de les 8 del matí fins a les 9 del vespre (horari d'obertura de la facultat). Degut a un error en l'execució del programa, entre els dies 5 i XX, només disposem de dades a partir de les  11:44.
L'API del racó ens proporciona la informació següent:
\begin{itemize}
	\item L'horari de classes de cada aula.
	\item El nombre d'ordinadors lliures per cada aula en l'instant en què es demana.
\end{itemize}

\subsection{Variables d'estudi}
Per tal de fer-nos una idea de com és la disponibilitat de les aules, hem decidit treballar amb les variables següents:
\begin{description}
	\item[X] = proporció d'ordinadors lliures en les aules on \emph{hi ha classe}.
	\item[Y] = proporció d'ordinadors lliures en les aules on \emph{no hi ha classe}.
\end{description}

\subsection{Contrast d'hipòtesis}
$$H0: \mu_x = \mu_y$$
$$H1: \mu_x \neq \mu_y$$

\subsection{Premisses}
\begin{enumerate}
	\item Les variables aleatòries \emph{X} i \emph{Y} es poden aproximar a una normal degut a la seva grandària (11934 i 11951, respectivament).
	\item Les diverses mostres preses de \emph{X} i \emph{Y} són consecutives i per tant no independents. Tanmateix, sí que podem suposar que al cap d'una hora d'haver pres una mostra, la nova mostra ja serà independent de l'anterior (suposem que hi ha prou moviment a les aules). Així doncs, si agaféssim només una de cada 60 mostres (1 cada hora), tindríem poblacions mostrals de, aproximadament, 200 individus, que són prou grans com per poder dir que \emph{X} i \emph{Y} s'aproximen a una normal (vegeu figures 1 i 2).
\end{enumerate}

\subsection{Estadístic utilitzat}
$$\hat{z} = \frac{\bar{x} - \bar{y}}{\sqrt{s_x^2/n_x + s_y^2/n_y}} $$
$$\hat{z} ~ N(0,1)$$


