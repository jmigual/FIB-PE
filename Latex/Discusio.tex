\section{Discussió}
\subsection{Prefaci}
El comportament de les variables \emph{X} o \emph{Y} pot dependre de molts factors. Tot seguit n'enumerem uns quants:
\begin{description}
	\item[Hora:] els hàbits rutinaris dels estudiants i els horaris de classe fan que en certes hores hi hagi més gent utilitzant els ordinadors.
	\item[Ocupació de les aules per fer classe:] com més aules estiguin reservades per fer-hi classe, menys ordinadors lliures hi haurà pels estudiants que no tenen classe i en volen fer servir un.
	\item[Exàmens:] els exàmens amb ordinador, a part d'ocupar moltes aules, poden provocar que hi hagi molts estudiants repassant a l'últim moment i que l'ocupació en les hores abans sigui anormalment alta.
\end{description}

\subsection{Conclusió principal}
Hem obtingut que la mitjana de disponibilitat de les aules amb classe és inferior a les mitjana de les aules sense classe. Això contradiu la nostra hipòtesis inicial del treball ja que crèiem que les aules amb classe tindrien una disponibilitat superior a les sense classe. 


